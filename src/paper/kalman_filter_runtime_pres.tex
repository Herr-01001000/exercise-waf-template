\documentclass[11pt]{beamer}
% \documentclass[11pt,handout]{beamer}
\usepackage[T1]{fontenc}
\usepackage[utf8]{inputenc}
\usepackage{float, afterpage, rotating, graphicx}
\usepackage{epstopdf}
\usepackage{longtable, booktabs, tabularx}
\usepackage{fancyvrb, moreverb, relsize}
\usepackage{eurosym, calc}
\usepackage{amsmath, amssymb, amsfonts, amsthm, bm} 



\usepackage{natbib}
\bibliographystyle{rusnat}


\hypersetup{colorlinks=true, linkcolor=black, anchorcolor=black, citecolor=black, filecolor=black, menucolor=black, runcolor=black, urlcolor=black}

\setbeamertemplate{footline}[frame number]
\setbeamertemplate{navigation symbols}{}
\setbeamertemplate{frametitle}{\centering\vspace{1ex}\insertframetitle\par}


\begin{document}

\title{Relation between Number of Observations and Runtime}

\author{Group 7: Wenxin Hu, Ruizhuo Wan, Yan Zhang, Xiaoran Wu, Tianfan Sun}


\begin{frame}
    \titlepage
    \note{~}
\end{frame}


\begin{frame}
        Introduction
\end{frame}


\begin{frame}
        In past assignments we realized Kalman filter with python and tried with multiple
        methods to achieve calculation speedups. In last assignment the most optimized speed-up was achieved with numba by 714 times. \\

        In this assignment we look into the relationships of optimized runtime with the size of the dataset.
\end{frame}


\begin{frame}
        Method
\end{frame}


\begin{frame}
        The process:
        \begin{enumerate}
                \item Experiment with polynomial orders of 1, 2, 3, 4, 5
                \item Create a dataframe of 2 columns: number of observables and runtimes
                \item Plot
        \end{enumerate}
\end{frame}


\begin{frame}
        The polynomial regressions are as follows:\\
        \begin{align*}
    \text{runtime} = &\alpha + \beta_1 \text{numbers of observations} \\
    \text{runtime} = &\alpha + \beta_1 \text{numbers of observations} + \beta_2 \text{numbers of observations}^2 \\
    \text{runtime} = &\alpha + \beta_1 \text{numbers of observations} + \beta_2 \text{numbers of observations}^2 \\ 
                     &+ \beta_3\text{numbers of observations}^3 \\
    \text{runtime} = &\alpha + \beta_1 \text{numbers of observations} + \beta_2 \text{numbers of observations}^2 \\
                     &+ \beta_3\text{numbers of observations}^3 + \beta_4\text{numbers of observations}^4 \\
    \text{runtime} = &\alpha + \beta_1 \text{numbers of observations} + \beta_2 \text{numbers of observations}^2 \\
                     &+ \beta_3\text{numbers of observations}^3 + \beta_4\text{numbers of observations}^4 \\
                     &+ \beta_5text{numbers of observations}^5
\end{align*}
\end{frame}


\begin{frame}
        \begin{figure}
    \caption{Regression plot of polynomial order 1}
    \includegraphics[width=\textwidth]{../../out/figures/timing_baseline_order_1}
        \end{figure}
\end{frame}


\begin{frame}
        \begin{figure}
    \caption{Regression plot of polynomial order 2}
    \includegraphics[width=\textwidth]{../../out/figures/timing_baseline_order_2}
        \end{figure}
\end{frame}


\begin{frame}
        \begin{figure}
    \caption{Regression plot of polynomial order 3}
    \includegraphics[width=\textwidth]{../../out/figures/timing_baseline_order_3}
        \end{figure}
\end{frame}


\begin{frame}
        \begin{figure}
    \caption{Regression plot of polynomial order 4}
    \includegraphics[width=\textwidth]{../../out/figures/timing_baseline_order_4}
        \end{figure}
\end{frame}


\begin{frame}
        \begin{figure}
    \caption{Regression plot of polynomial order 5}
    \includegraphics[width=\textwidth]{../../out/figures/timing_baseline_order_5}
        \end{figure}
\end{frame}


\begin{frame}
Conclusion
\end{frame}


\begin{frame}
        As number of observations became bigger, the runtime slowed down and variated in each running. It is of same tendency with different numbers of polynomials.
\end{frame}


\begin{frame}[t]
    \frametitle{Technical facts}
    \begin{itemize}
        \item<+-> Template cited from: \citet{GaudeckerEconProjectTemplates}
        \item<+-> Former realization of Kallman filters was based on \citet{Gabler18}.
    \end{itemize}
    \note{~}
\end{frame}


% Print black screen only in presentation mode for finishing up.
\mode<beamer> {
    \beamersetaveragebackground{black}
    \begin{frame}
        \frametitle{}
    \end{frame}

    \beamersetaveragebackground{white}
}

\begin{frame}[allowframebreaks]
    \frametitle{References}
    
    \bibliography{refs}
    
\end{frame}

\end{document}